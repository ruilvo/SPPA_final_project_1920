% Contents\
\begin{abstract}
    Some abstract
\end{abstract}


\section{Exercise 1}\FloatBarrier

In this exercise, and for each of the three signals in study, I decided to test
three different filter orders (\(n\)), with three different adaptation steps
(\(\mu\)). These are:
\begin{equation}
    \begin{aligned}
        n   & = {3, 5, 10}      \\
        \mu & = {0.3, 0.5, 0.9}
    \end{aligned}
\end{equation}

The experiment is done by script {\tt experiment1.m} and supported by the function
    {\tt do\_nlms.m}. Furthermore, plotting is handled by the Python script {\tt
        makeplots\_1.py}.

There is a lot to unpack here, so let's get started. The first signal a sequence
of random numbers that is filtered and to which is added some white noise
component. The filter is given by:
\begin{equation}
    F(Z) = \frac{Z+2Z^{-1}}{Z+0.5Z^{-1}}
\end{equation}
This filter has one pole and one zero. Figure~\ref{fig:ex1varystep} shows how the
adaptation evolves in regarding to the adaptation step. Keep in mind that, as
asked, the adaptation step goes down by a factor of 10 in the middle of the
adaptation.
\begin{figure}[h]
    \centering
    \begin{subfigure}[t]{0.30\columnwidth}
        \centering
        \includegraphics[width=\columnwidth]{pdf/ex1_l1_n5_mu90.pdf}
        \caption{}
    \end{subfigure} \hfill
    \begin{subfigure}[t]{0.30\columnwidth}
        \centering
        \includegraphics[width=\columnwidth]{pdf/ex1_l1_n5_mu60.pdf}
        \caption{}
    \end{subfigure} \hfill
    \begin{subfigure}[t]{0.30\columnwidth}
        \centering
        \includegraphics[width=\columnwidth]{pdf/ex1_l1_n5_mu30.pdf}
        \caption{}
    \end{subfigure}
    \caption{Results for the first signal of exercise 1: varying adaptation
        step.\label{fig:ex1varystep}}
\end{figure}
We see nothing too extraordinary: with smaller adaptation steps, the error is
smaller. This is the intuitive results, however, one must note that this might not
always be the case: a slow adaptation step can cause problems on dynamic systems,
as the adaptive filter might not adapt fast enough. Also, with smaller adaptation
steps, the level of error (which is equivalent to a noise floor of the system) is
not only lower but less jittery. We can think of this optimization problem as
looking for solutions in a convex space: smaller adaptation steps correspond to a
smaller oscillation in the bottom of the concave surface. We see that with a
filter order of 3 the noise floor is between \SI{-24}{\decibel} to
\SI{-26}{\decibel}. Of course that the next step is to compare this floor in
regard to the filter order. Figure~\ref{fig:ex1varyorder} shows this.
\begin{figure}[h]
    \centering
    \begin{subfigure}[t]{0.30\columnwidth}
        \centering
        \includegraphics[width=\columnwidth]{pdf/ex1_l1_n3_mu60.pdf}
        \caption{}
    \end{subfigure} \hfill
    \begin{subfigure}[t]{0.30\columnwidth}
        \centering
        \includegraphics[width=\columnwidth]{pdf/ex1_l1_n5_mu60.pdf}
        \caption{}
    \end{subfigure} \hfill
    \begin{subfigure}[t]{0.30\columnwidth}
        \centering
        \includegraphics[width=\columnwidth]{pdf/ex1_l1_n10_mu60.pdf}
        \caption{}
    \end{subfigure}
    \caption{Results for the first signal of exercise 1: varying filter
        order.\label{fig:ex1varyorder}}
\end{figure}
We see that for the same adaptation step, increasing the filter order decreases
both the error of the adaptation and the fluctuations of the weights of the
filter.

With the second signal, things get more interesting. In
figure~\ref{fig:ex1sig2switch}, we can see an interesting effect: in the first
sinusoidal section, the weight marked in blue has a larger value than the one in
orange, situation that inverts in the second case. Since a sinusoidal is fully
described by amplitude and phase, the system is overdetermined. The converge to
specific set of weights is not deterministic. We observe the adaptation step is
the most relevant factor in the sinusoidal regions and that the number of weights
becomes more important in the other regions.
\begin{figure}
    \centering
    \begin{subfigure}[t]{0.30\columnwidth}
        \centering
        \includegraphics[width=\columnwidth]{pdf/ex1_l2_n3_mu90.pdf}
        \caption{\label{fig:ex1sig2switch}}
    \end{subfigure} % \hspace{1cm}
    \begin{subfigure}[t]{0.30\columnwidth}
        \centering
        \includegraphics[width=\columnwidth]{pdf/ex1_l2_n3_mu60.pdf}
        \caption{}
    \end{subfigure} \\
    \begin{subfigure}[t]{0.30\columnwidth}
        \centering
        \includegraphics[width=\columnwidth]{pdf/ex1_l2_n5_mu90.pdf}
        \caption{}
    \end{subfigure} % \hspace{1cm}
    \begin{subfigure}[t]{0.30\columnwidth}
        \centering
        \includegraphics[width=\columnwidth]{pdf/ex1_l2_n5_mu60.pdf}
        \caption{}
    \end{subfigure}
    \caption{Results for the second signal of exercise 1.\label{fig:ex1sig2}}
\end{figure}

The last signal is probably the most interesting one to analyse.
Figure~\ref{fig:ex1vsig3} depicts the results for this exercise.
\begin{figure}[h]
    \centering
    \begin{subfigure}[t]{0.30\columnwidth}
        \centering
        \includegraphics[width=\columnwidth]{pdf/ex1_l3_n3_mu90.pdf}
        \caption{}
    \end{subfigure} \hfill
    \begin{subfigure}[t]{0.30\columnwidth}
        \centering
        \includegraphics[width=\columnwidth]{pdf/ex1_l3_n3_mu60.pdf}
        \caption{}
    \end{subfigure} \hfill
    \begin{subfigure}[t]{0.30\columnwidth}
        \centering
        \includegraphics[width=\columnwidth]{pdf/ex1_l3_n3_mu30.pdf}
        \caption{}
    \end{subfigure} \\
    \begin{subfigure}[t]{0.30\columnwidth}
        \centering
        \includegraphics[width=\columnwidth]{pdf/ex1_l3_n5_mu90.pdf}
        \caption{}
    \end{subfigure} \hfill
    \begin{subfigure}[t]{0.30\columnwidth}
        \centering
        \includegraphics[width=\columnwidth]{pdf/ex1_l3_n5_mu60.pdf}
        \caption{}
    \end{subfigure} \hfill
    \begin{subfigure}[t]{0.30\columnwidth}
        \centering
        \includegraphics[width=\columnwidth]{pdf/ex1_l3_n5_mu30.pdf}
        \caption{}
    \end{subfigure} \\
    \begin{subfigure}[t]{0.30\columnwidth}
        \centering
        \includegraphics[width=\columnwidth]{pdf/ex1_l3_n10_mu90.pdf}
        \caption{}
    \end{subfigure} \hfill
    \begin{subfigure}[t]{0.30\columnwidth}
        \centering
        \includegraphics[width=\columnwidth]{pdf/ex1_l3_n10_mu60.pdf}
        \caption{}
    \end{subfigure} \hfill
    \begin{subfigure}[t]{0.30\columnwidth}
        \centering
        \includegraphics[width=\columnwidth]{pdf/ex1_l3_n10_mu30.pdf}
        \caption{\label{fig:ex1vsig3n10mu3}}
    \end{subfigure}
    \caption{Results for the third signal of exercise . The black dashed lines on
        the second subplot in each figure corresponds to the theoretical weights.
        \label{fig:ex1vsig3}}
\end{figure}
The way the figures are displayed, vertically we can see the effect of changing
the number of weights and horizontally the effect of improving the adaptation
step. We can see that for the coarser adaptation step, there is a lot of jittering
in the weights every time there is an abrupt change in the weights. In a way, we
can say the weights lost ``synchronization'' with the system and it needs to
recover. The behaviour is similar to a weakly damped oscillation after an impulse
impulse, hinting about the stability of the algorithm. On the other hand, on the
finer adaptation steps, two effects become clearly visible. One is that the
adaptation clearly starts to lag behind the ideal weights. The second, especially
visible in the weight that varies as a square wave, is that there is a clear
low-pass filter effect. This can get really extreme. I'd like to point out the
particular case of the figure~\label{fig:ex1vsig3n10mu3} for \(n=10\) and
\(\mu=0.3\) where we can see that in the first part the error is actually lower
than in the second part, where \(\mu=0.03\). The effect of an adaptation that is
too slow becomes clearly visible with the decrease in performance.

\FloatBarrier
\section{Exercise 2}

In exercise 2, we are to consider three signals:
\begin{align}
    x_1 (t) & = v(t) + 3v(t-1) + 3v(t-2)  \\
    x_2 (t) & = 3v(t) + 3v(t-1) + 1v(t-2) \\
    d(t)    & = v(t) - 2v(t-1) + v(t-2)
\end{align}
where \(v(t)\) is a Gaussian white noise stochastic process with zero mean and
unit variance.

In the first part of the exercise, we are asked to compute the autocorrelation of
each \(x_i\) signal and the cross-correlation between \(x_i\) and \(d\) for each
\(i\). I will present the process for one of them, since the other ones are
analogous. I'll do the autocorrelation of \(x_1\), denoted \(r_{x_1x_1}(k)\):
\begin{equation}
    r_{x_1x_1}(k) = \left< x_1 (t+k), x_1(t) \right>
\end{equation}
The first step is to expand \(x_1\):
\begin{equation}
    r_{x_1x_1}(k) = \left< v(t+k) + 3v(t-1+k) + 3v(t-2+k),
    v(t) + 3v(t-1) + 3v(t-2) \right>
\end{equation}
The inner product operation has the distributive property, so this is equivalent
to:
\begin{equation}
    \begin{aligned}
        r_{x_1x_1}(k) & = \left< v(t+k), v(t) \right>        \\
                      & + \left< v(t+k), 3v(t-1) \right>     \\
                      & + \left< v(t+k),  3v(t-2) \right>    \\
                      & + \left< 3v(t-1+k), v(t) \right>     \\
                      & + \left< 3v(t-1+k), 3v(t-1) \right>  \\
                      & +\left< 3v(t-1+k),  3v(t-2) \right>  \\
                      & + \left< 3v(t-2+k), v(t) \right>     \\
                      & + \left< 3v(t-2+k), 3v(t-1) \right>  \\
                      & + \left< 3v(t-2+k),  3v(t-2) \right> \\
    \end{aligned}
\end{equation}
Now we should pay close attention to this expression. Since \(v(t)\) is a Gaussian
stochastic process:
\begin{equation}
    \left<v(t-k),v(t-n)\right>=\begin{cases}
        1 & \text{if~} k=n   \\
        0 & \text{otherwise}
    \end{cases}
\end{equation}
Which massively simplifies the problem. For example, for \(k=-1\), only the second
and sixth parcels contribute to the result. And on these, it's just a matter of
multiplying the weights of the parcels together and then sum them.

We are also asked to confirm the results with simulation data. For that, I created
the MATLAB script {\tt experiment21.m}. It uses MATLAB's filter function to apply
the \(k\)-th delay to the signal. Since it's not possible to do such filter for
positive \(k\), what I do is delay the other signal instead.
Table~\ref{tab:results21} shows both the analytical and simulated results, as well
as the relative error for the simulated data.

\begin{table}
    \adjustbox{width=\columnwidth,keepaspectratio}{
        \begin{tabular}{|c|c|c|c|c|c|c|c|c|c|c|c|c|c|c|c|c|c|c|c|c|}
    \hline
    \multirow{3}{*}{\(k\)} & \multicolumn{5}{c|}{\(r_{x_1x_1}\)} & \multicolumn{5}{c|}{\(r_{x_1d}\)} & \multicolumn{5}{c|}{\(r_{x_2x_2}\)} & \multicolumn{5}{c|}{\(r_{x_2d}\)}\tabularnewline
    \cline{2-21} \cline{3-21} \cline{4-21} \cline{5-21} \cline{6-21} \cline{7-21} \cline{8-21} \cline{9-21} \cline{10-21} \cline{11-21} \cline{12-21} \cline{13-21} \cline{14-21} \cline{15-21} \cline{16-21} \cline{17-21} \cline{18-21} \cline{19-21} \cline{20-21} \cline{21-21}
     & \multirow{2}{*}{Analyt.} & \multicolumn{4}{c|}{Sim.} & \multirow{2}{*}{Analyt.} & \multicolumn{4}{c|}{Sim.} & \multirow{2}{*}{Analyt.} & \multicolumn{4}{c|}{Sim.} & \multirow{2}{*}{Analyt.} & \multicolumn{4}{c|}{Sim.}\tabularnewline
    \cline{3-6} \cline{4-6} \cline{5-6} \cline{6-6} \cline{8-11} \cline{9-11} \cline{10-11} \cline{11-11} \cline{13-16} \cline{14-16} \cline{15-16} \cline{16-16} \cline{18-21} \cline{19-21} \cline{20-21} \cline{21-21}
     &  & \(n=n_1\) & rel.\ error (\%) & \(n=n_2\) & rel.\ error (\%) &  & \(n=n_1\) & rel.\ error (\%) & \(n=n_2\) & rel.\ error (\%) &  & \(n=n_1\) & rel.\ error (\%) & \(n=n_2\) & rel.\ error (\%) &  & \(n=n_1\) & rel.\ error (\%) & \(n=n_2\) & rel.\ error (\%)\tabularnewline
    \hline
    \(-3\) & \(0\) & \(0.00740\) &  & \(-0.00138\) &  & \(0\) & \(0.01074\) &  & \(0.00051\) &  & \(0\) & \(0.01948\) &  & \(-0.00004\) &  & \(0\) & \(-0.00062\) &  & \(0.00000\) & \tabularnewline
    \hline
    \(-2\) & \(3\) & \(3.04359\) & \(14.53\) & \(3.00077\) & \(0.05\) & \(1\) & \(1.00635\) & \(6.35\) & \(0.99961\) & \(0.06\) & \(3\) & \(3.01627\) & \(5.42\) & \(3.00004\) & \(0.01\) & \(3\) & \(3.00270\) & \(0.90\) & \(2.99994\) & \(0.06\)\tabularnewline
    \hline
    \(-1\) & \(12\) & \(12.05903\) & \(4.92\) & \(12.00114\) & \(0.23\) & \(1\) & \(0.98318\) & \(16.82\) & \(0.99962\) & \(0.02\) & \(12\) & \(12.01831\) & \(1.53\) & \(12.00006\) & \(0.04\) & \(-3\) & \(-3.00103\) & \(-0.34\) & \(-2.99988\) & \(-0.36\)\tabularnewline
    \hline
    \(0\) & \(19\) & \(19.04664\) & \(2.45\) & \(19.00069\) & \(0.28\) & \(-2\) & \(-1.99454\) & \(-2.73\) & \(-1.99976\) & \(-0.09\) & \(19\) & \(19.02250\) & \(1.18\) & \(19.00009\) & \(0.07\) & \(-2\) & \(-2.00287\) & \(-1.44\) & \(-2.00021\) & \(-0.15\)\tabularnewline
    \hline
    \(1\) & \(12\) & \(12.05903\) & \(4.92\) & \(12.00114\) & \(0.23\) & \(-3\) & \(-2.99158\) & \(-2.81\) & \(-2.99982\) & \(-0.06\) & \(12\) & \(12.01831\) & \(1.53\) & \(12.00006\) & \(0.04\) & \(1\) & \(1.00332\) & \(3.32\) & \(1.00021\) & \(0.06\)\tabularnewline
    \hline
    \(2\) & \(3\) & \(3.04359\) & \(14.53\) & \(3.00077\) & \(0.05\) & \(3\) & \(2.98048\) & \(6.51\) & \(2.99947\) & \(0.08\) & \(3\) & \(3.01627\) & \(5.42\) & \(3.00004\) & \(0.01\) & \(1\) & \(1.00081\) & \(0.81\) & \(1.00000\) & \(0.00\)\tabularnewline
    \hline
    \(3\) & \(0\) & \(0.00740\) &  & \(-0.00138\) &  & \(0\) & \(0.00979\) &  & \(-0.00013\) &  & \(0\) & \(0.01948\) &  & \(-0.00004\) &  & \(0\) & \(-0.00716\) &  & \(-0.00068\) & \tabularnewline
    \hline
    \end{tabular}

    }
    \caption{Results for the first part of exercise 2. \(n_1=1000000\)
        and \(n_2=ns = 100000000\)\label{tab:results21}}
\end{table}

The second part of the exercise asks us to solve the system of normal equations
for various number of weights and to plot the error as function of this. Results
for this can be seen in figures~\ref{fig:ex22x1} and~\ref{fig:ex22x2}.
\begin{figure}
    \centering
    \begin{minipage}[t]{0.30\columnwidth}
        \centering
        \includegraphics[width=\columnwidth]{pdf/ex2_pt2_x1_all.pdf}
        \caption{Results for the second part of ex. 2 for \(x_1(t)\)
            \label{fig:ex22x1}}
    \end{minipage}
    \hspace{1cm}
    \begin{minipage}[t]{0.30\columnwidth}
        \centering
        \includegraphics[width=\columnwidth]{pdf/ex2_pt2_x2_all.pdf}
        \caption{Results for the second part of ex. 2 for \(x_2(t)\)
            \label{fig:ex22x2}}
    \end{minipage}
\end{figure}
We see immediately that for the \(x_1\) the error never leaves the
\SI{7}{\decibel} range, regardless of the filter order. On the other hand, for
\(x_2\) we see a clear convergence to around \SI{-150}{\decibel}, coherent with
limitations of floating point arithmetic.
% To understand why, let's look at the characteristics of \(x_1\) and \(x_2\). We
% can say \(x_1(t)\) is obtained from \(v(t)\) by a filter of the form \(Z +
% 3Z^{-1} + 3Z^{-2}\) and \(x_2(t)\) from a filter of the form \(3Z + 3Z^{-1} +
% Z^{-2}\). The first one has roots outside the unit circle while the roots of the
% later one are inside the unit circle. Having the zeros outside the unit circle
% means one of two things: if we choose as the region of convergence (ROC) the
% inner circle, we have a system that is stable, but is anti-causal. On the other
% hand, if we choose as the ROC the external region of the Z-plane the system
% becomes causal, but at the cost of stability.

\FloatBarrier
\section{Exercise 2}

In exercise 3, we are asked to use the NMLS algorithm to make an adaptive filter
that works with the previous signals. For this, I sidestepped a bit from script.
The number of iterations and the step were both too low to see the best converge
possible. The same with the number of weights. I settled on two million
iterations, with \(\mu=0.1\) and \(40\) weights. Results can be seen in
figure~\ref{fig:ex3res}.
\begin{figure}
    \centering
    \begin{subfigure}[t]{0.30\columnwidth}
        \centering
        \includegraphics[width=\columnwidth]{pdf/ex3_x1_all.pdf}
        \caption{Results for \(x_1\).}
    \end{subfigure} \hspace{1cm}
    \begin{subfigure}[t]{0.30\columnwidth}
        \centering
        \includegraphics[width=\columnwidth]{pdf/ex3_x2_all.pdf}
        \caption{Results for \(x_2\).}
    \end{subfigure}
    \caption{Results for the exercise 3.\label{fig:ex3res}}
\end{figure}
