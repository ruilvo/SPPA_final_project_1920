% Basic geometry
\usepackage[a4paper,margin=3cm]{geometry}

% Macros specific for pdftex / luatex / xetex compilers
\usepackage{ifthen}
\usepackage{ifxetex,ifluatex}

% Compiler stuff
% For LuaLaTeX and XeLaTeX
\newcommand{\compilersettingsluaxe}{
    % Put here things that are common to XeLaTeX and LuaLaTeX
    \usepackage{fontspec}
  }
\ifluatex
    \typeout{ *** LuaLaTeX *** ^^J}
    % LuaLaTeX specific code
    \compilersettingsluaxe
    % \usepackage{luatextra}
    % \usepackage[utf8]{luainputenc} % Only needed for old fonts
\else\ifxetex
    \typeout{ *** XeLaTeX *** ^^J}
    % XeLaTeX specific code
    \compilersettingsluaxe
\else
    \typeout{ *** LaTeX *** ^^J}
    % pdfLaTeX specific code
    \usepackage[utf8]{inputenc}
    \usepackage[T1]{fontenc}
\fi\fi

% My fonts
% For LuaLaTeX and XeLaTeX
\newcommand{\fontsettingsluaxe}{
    % Put here things that are common to XeLaTeX and LuaLaTeX
    \setsansfont{Arial}  % Closest thing in Win to Helvetica
  }
\ifluatex
    % LuaLaTeX specific code
    \fontsettingsluaxe
\else\ifxetex
    % XeLaTeX specific code
    \fontsettingsluaxe
\else
    % pdfLaTeX specific code
    \usepackage[scaled]{helvet}
\fi\fi

% Set the sans-serifs font as the main font.
% Comment this out if you don't want this behaviour.
\renewcommand*\familydefault{\sfdefault}

% Language hyphenation and typographical rules
\usepackage[portuguese,english]{babel}
\usepackage{csquotes}

% Multiple authors and affiliations
\usepackage{authblk}

% Code listings
\usepackage{listings}

% Bibliography setup
% https://www.overleaf.com/learn/latex/Articles/Getting_started_with_BibLaTeX
% https://www.overleaf.com/learn/latex/Bibliography_management_with_biblatex
% https://www.overleaf.com/learn/latex/Bibtex%20bibliography%20styles#Biblatex_styles
% https://www.overleaf.com/learn/latex/Natbib%20citation%20styles#Biblatex_citation_styles
\usepackage[style=ieee,
    natbib=true,
    backend=biber,
    bibencoding=utf8,
    doi=false,
    isbn=false,
    url=false,
    maxnames=3,
    backref=true]
        {biblatex}
% Citing by section might involve re-writing the macro
% https://tex.stackexchange.com/questions/208678/biblatex-customizing-backreferences
\DefineBibliographyStrings{english}{%
backrefpage = {cited on page},% originally "cited on page"
backrefpages = {cited on pages},% originally "cited on pages"
}
\DefineBibliographyStrings{portuguese}{%
backrefpage = {citado na página},
backrefpages = {citado nas páginas},
}

% Tables stuff
\usepackage{array}
\usepackage{multirow}

% Interesting float placements (like 'H') and custom float types
\usepackage{float}
\usepackage{placeins}
% Use pgf pictures
\usepackage{pgf}
\usepackage{adjustbox}
% Place floats *above* footnotes
\usepackage[bottom,symbol]{footmisc}
% Reset footnote counter every page
\usepackage{perpage}
\MakePerPage{footnote}
% Set default float placement
% \makeatletter
% \renewcommand{\fps@figure}{tbph}
% \renewcommand{\fps@table}{tbph}
% \makeatother

% Pretty colours
\usepackage{xcolor}

% Graphics stuff
\usepackage{graphicx}
\graphicspath{{figures/}}

% SVGs with Inkscape and PDF+LaTeX
% https://tex.stackexchange.com/questions/473994/svg-and-inkscape
\usepackage[inkscapearea=page]{svg}
\svgpath{{svg/}}

% TikZ pictures
\usepackage{tikz}
\usepackage{pgfplots}
\pgfplotsset{compat=newest}
\usepgfplotslibrary{groupplots}
\usepgfplotslibrary{dateplot}

% For sub-figures and stuff
\usepackage{caption}
\usepackage{subcaption}

% Math stuff
\usepackage{amsmath} % Interesting environments
\usepackage{commath} % Interesting macros
\usepackage{amssymb} % Interesting symbols

\usepackage{siunitx} % Units and numbers in text

% Use more than one optional parameter in a new commands
\usepackage{xargs}

% Flush two column page ends
%\usepackage{flushend}
\usepackage{balance}

% Interesting URL breakings
\usepackage{url}
\def\UrlBreaks{\do\/\do-\do\&\do.\do:}

% Hyperref and Backref
% backref makes the bibliography say where the entry was cited.
% For the print version of the thesis you might wanna set all colors to back
\usepackage{hyperref}
\hypersetup{colorlinks,citecolor=blue,urlcolor=blue,linkcolor=blue, breaklinks=true}

% Notes on the documents
% https://tex.stackexchange.com/questions/9796/how-to-add-todo-notes
% https://tex.stackexchange.com/questions/316220/todo-commentsnot-include-and-left-align
% Examples:
% \unsure{Is this correct?}, \change{Change this!},
% \info{This can help me in chapter seven!}
% \improvement{This really needs to be improved!\\ What was I thinking?!}
% \thiswillnotshow{This is hidden since option `disable' is chosen!}
% WARNING: It eliminates whitespaces in front of it.
% You can add trailing {} to avoid.

\usepackage[colorinlistoftodos,prependcaption,textsize=tiny, textwidth=2cm]{todonotes}
\newcommandx{\unsure}[2][1=]{
    \todo[linecolor=red,backgroundcolor=red!25,bordercolor=red,#1]{#2}
    }
\newcommandx{\change}[2][1=]{
    \todo[linecolor=blue,backgroundcolor=blue!25,bordercolor=blue,#1]{#2}
    }
\newcommandx{\info}[2][1=]{
    \todo[linecolor=green,backgroundcolor=green!25,bordercolor=green,#1]{#2}
    }
\newcommandx{\improvement}[2][1=]{
    \todo[linecolor=yellow,backgroundcolor=yellow!25,bordercolor=yellow,#1]{#2}
    }
\newcommandx{\thiswillnotshow}[2][1=]{\todo[disable,#1]{#2}}
